% Аннотация на русском языке
\pdfbookmark{АННОТАЦИЯ}{Annotation}
\cchapter{АННОТАЦИЯ}                       % Заголовок

% Число рисунков, таблиц, источников, приложений и общее 
% количество страниц ВКР подсчитываются автоматически.

Выпускная квалификационная работа изложена на  
\formbytotal{TotPages}{страниц}{е}{ах}{ах},
содержит
\formbytotal{totalcount@figure}{рисун}{ок}{ка}{ков},
\formbytotal{totalcount@table}{таблиц}{у}{ы}{},
\formbytotal{citenum}{источник}{}{а}{ов}, 
\formbytotal{totalappendix}{приложени}{е}{я}{й}.
\bigskip

\noindent
КЛЮЧЕВЫЕ СЛОВА, КЛЮЧЕВЫЕ СЛОВА, КЛЮЧЕВЫЕ СЛОВА, КЛЮЧЕВЫЕ СЛОВА, 
КЛЮЧЕВЫЕ СЛОВА
\bigskip

Перечень ключевых слов должен характеризовать содержание реферируемой 
ВКР. Он должен включать до пяти ключевых слов в~именительном падеже, 
напечатанных последовательно через запятые. 

Текст аннотации, помимо сведений об объёме ВКР и ключевых слов, 
включает: сущность выполненной работы (её цель, объект исследования), 
описание методов исследования и аппаратуры; конкретные сведения, 
раскрывающие содержание основной части ВКР; краткие выводы 
об особенностях работы, её эффективности, возможности и области 
применения полученных результатов, их новизну. Каждая фраза аннотации 
должна быть носителем информации. Аннотация не должна подменять 
оглавления и должен быть достаточно полным. Объём аннотации "--- 
не более одной страницы.


% Аннотация на английском языке
\clearpage
\pdfbookmark{ABSTRACT}{Abstract}
\cchapter{ABSTRACT}                       % Заголовок


The Bachelor's thesis has
\formbytotalen{TotPages}{page}{}{s},
\formbytotalen{totalcount@figure}{figure}{}{s},
\formbytotalen{totalcount@table}{table}{}{s},
\formbytotalen{citenum}{reference}{}{s}, 
\formbytotalen{totalappendix}{appendi}{x}{cies}.
\bigskip

\noindent
KEYWORD, KEYWORD, KEYWORD, KEYWORD, KEYWORD
\bigskip

As any dedicated reader can clearly see, the Ideal of
practical reason is a representation of, as far as I know, the things
in themselves; as I have shown elsewhere, the phenomena should only be
used as a canon for our understanding. The paralogisms of practical
reason are what first give rise to the architectonic of practical
reason. As will easily be shown in the next section, reason would
thereby be made to contradict, in view of these considerations, the
Ideal of practical reason, yet the manifold depends on the phenomena.

Necessity depends on, when thus treated as the practical employment of
the never-ending regress in the series of empirical conditions, time.
Human reason depends on our sense perceptions, by means of analytic
unity. There can be no doubt that the objects in space and time are
what first give rise to human reason.

