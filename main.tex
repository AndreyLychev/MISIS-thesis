\documentclass[
    bachelor,
    blackurls,
    titleikn
]{mthesis}

%%%%%%%%%%%%%%%%%%%%%%%%%%%%%%%%%%%%%%%%%%%%%%%%%%%%%%%%%%%%%%%%%%%
% Параметры класса документа

%---------
% тип ВКР
%---------
% bachelor  - бакалаврская диссертация
% master    - магистерская диссертация

%-------------
% гиперссылки
%-------------
% blackurls - ссылки на литературу, таблицы, рисунки, формулы и т.п.
%             не выделяются цветом (используется для ч/б печати)
% colurls   - цветные ссылки (для распространения пояснительной записки 
%             в электронном виде)

%-----------------------------
% оформление титульного листа
%-----------------------------
% titlestd  - согласно методическим указаниям
% titleikn  - согласно требованиям ИКН


%%%%%%%%%%%%%%%%%%%%%%%%%%%%%%%%%%%%%%%%%%%%%%%%%%%%%%%%%%%%%%%%%%%%
% Здесь следует размещать допольнительные пакеты и команды. Следующие 
% пакеты уже загружены в стилевой файл: geometry, amssymb, mathtools, 
% ifthen, cmap, textcomp, babel, upgreek, indentfirst, xcolor, calc, 
% tabularray, icomma, ulem, soulutf8, tikz, hyperref, graphicx, 
% caption, subcaption, pdfpages, aliascnt, totalcount, totcount, 
% totpages, placeins, enumitem, biblatex, fancyvrb, listings



%%%%%%%%%%%%%%%%%%%%%%%%%%%%%%%%%%%%%%%%%%%%%%%%%%%%%%%%%%%%%%%%%%%%
% Информация для заполнения титульного листа и задания на ВКР

% ФИО студента
\AuthorShortName{Иванов И.И.}
\AuthorFullName{Иванов Иван Иванович} % полностью
\AuthorFullNameGen{Иванову Ивану Ивановичу} % полностью в родительном падеже
% Номер группы
\AuthorGroup{БИВТ-18-2}

% Тема работы
\ThesisTitle{Название темы выпускной квалификационной работы}

% Научный руководитель
\SupervisorShort{Петров П.П.}
\SupervisorFull{Петров Петр Петрович}
\SupervisorDegree{Доцент кафедры АСУ, к.т.н., доц.}

% Нормоконтроль
\GOSTComplianceReviewer{Смирнов С.С.}
% Проверка на заимствования
\PlagiarismChecker{Кузнецов К.К.}

% Название кафедры, ФИО зав. кафедрой
\DeptShort{АСУ}
\DeptFull{АВТОМАТИЗИРОВАННЫХ СИСТЕМ УПРАВЛЕНИЯ}
\DeptHead{Темкин И.О.}

% Название института, ФИО директора
\InstShort{ИТКН}
\InstFull{ИНФОРМАЦИОННЫХ ТЕХНОЛОГИЙ И КОМПЬЮТЕРНЫХ НАУК}
\InstHead{Солодов С.В.}

% Шифр специальности, краткое и полное наименование специальности
\SpecShort{09.03.01 ИВТ}
\SpecFull{09.03.01 ИНФОРМАТИКА И ВЫЧИСЛИТЕЛЬНАЯ ТЕХНИКА}

% Место и год выполнения работы
\City{Москва}
\Year{2033}

% Цель работы
\ThesisPurpose{}

% Исходные данные
\ThesisData{}
 
% Основная литература, в том числе:
% Монографии, учебники и т.п.
\ThesisBooks{}

% Отчеты по НИР, диссертации, дипломные работы и т.п.
\ThesisReports{}

% Периодическая литература (журналы)
\ThesisJournals{}

% Справочники и методическая литература (в том числе литература 
% по методам обработки экспериментальных данных)
\ThesisManuals{}

% Перечень основных этапов исследования и форма промежуточной 
% отчётности по каждому этапу
\ThesisStages{}

% Аппаратура и методики, которые должны быть использованы в работе
\ThesisEquipment{}

% Использование ЭВМ
\ThesisComputer{}

% Перечень (примерный) основных вопросов, которые должны быть 
% рассмотрены и проанализированы в литературном обзоре
\ThesisLitReview{}

% Перечень (примерный) графического и иллюстрированного материала
\IllustrMaterials{}

% Дата выдачи задания
\DateAssignment{<<18>> декабря 2033\,г.}

% Дата утверждения задания (должна быть не раньше даты выдачи задания)
\DateApproval{<<22>> декабря 2033\,г.}



\begin{document}

% Реферат
% Аннотация на русском языке
\pdfbookmark{АННОТАЦИЯ}{Annotation}
\cchapter{АННОТАЦИЯ}                       % Заголовок

% Число рисунков, таблиц, источников, приложений и общее 
% количество страниц ВКР подсчитываются автоматически.

Выпускная квалификационная работа изложена на  
\formbytotal{TotPages}{страниц}{е}{ах}{ах},
содержит
\formbytotal{totalcount@figure}{рисун}{ок}{ка}{ков},
\formbytotal{totalcount@table}{таблиц}{у}{ы}{},
\formbytotal{citenum}{источник}{}{а}{ов}, 
\formbytotal{totalappendix}{приложени}{е}{я}{й}.
\bigskip

\noindent
КЛЮЧЕВЫЕ СЛОВА, КЛЮЧЕВЫЕ СЛОВА, КЛЮЧЕВЫЕ СЛОВА, КЛЮЧЕВЫЕ СЛОВА, 
КЛЮЧЕВЫЕ СЛОВА
\bigskip

Перечень ключевых слов должен характеризовать содержание реферируемой 
ВКР. Он должен включать до пяти ключевых слов в~именительном падеже, 
напечатанных последовательно через запятые. 

Текст аннотации, помимо сведений об объёме ВКР и ключевых слов, 
включает: сущность выполненной работы (её цель, объект исследования), 
описание методов исследования и аппаратуры; конкретные сведения, 
раскрывающие содержание основной части ВКР; краткие выводы 
об особенностях работы, её эффективности, возможности и области 
применения полученных результатов, их новизну. Каждая фраза аннотации 
должна быть носителем информации. Аннотация не должна подменять 
оглавления и должен быть достаточно полным. Объём аннотации "--- 
не более одной страницы.


% Аннотация на английском языке
\clearpage
\pdfbookmark{ABSTRACT}{Abstract}
\cchapter{ABSTRACT}                       % Заголовок


The Bachelor's thesis has
\formbytotalen{TotPages}{page}{}{s},
\formbytotalen{totalcount@figure}{figure}{}{s},
\formbytotalen{totalcount@table}{table}{}{s},
\formbytotalen{citenum}{reference}{}{s}, 
\formbytotalen{totalappendix}{appendi}{x}{cies}.
\bigskip

\noindent
KEYWORD, KEYWORD, KEYWORD, KEYWORD, KEYWORD
\bigskip

As any dedicated reader can clearly see, the Ideal of
practical reason is a representation of, as far as I know, the things
in themselves; as I have shown elsewhere, the phenomena should only be
used as a canon for our understanding. The paralogisms of practical
reason are what first give rise to the architectonic of practical
reason. As will easily be shown in the next section, reason would
thereby be made to contradict, in view of these considerations, the
Ideal of practical reason, yet the manifold depends on the phenomena.

Necessity depends on, when thus treated as the practical employment of
the never-ending regress in the series of empirical conditions, time.
Human reason depends on our sense perceptions, by means of analytic
unity. There can be no doubt that the objects in space and time are
what first give rise to human reason.



% Содержание
\ifdefmacro{\microtypesetup}{\microtypesetup{protrusion=false}}{}
\tableofcontents*
\ifdefmacro{\microtypesetup}{\microtypesetup{protrusion=true}}{}

% Список сокращений и условных обозначений
%\cchapter{ПЕРЕЧЕНЬ СОКРАЩЕНИЙ И ОБОЗНАЧЕНИЙ}                        % Заголовок
\addcontentsline{toc}{chapter}{ПЕРЕЧЕНЬ СОКРАЩЕНИЙ И ОБОЗНАЧЕНИЙ}   % Добавляем его в оглавление

В настоящей выпускной квалификационной работе применяются следующие 
сокращения и~обозначения:

\smallskip\noindent
\begin{tblr}{colspec={lX[l]}, vline{2}={text=\cyrdash{}}, 
             column{1}={leftsep=0pt}, rows={abovesep+=-1pt,belowsep=0pt}}
БП   & бизнес-процесс \\
ВКР  & выпускная квалификационная работа \\
ИБ   & информационная безопасность \\
ИИ   & искусственный интеллект \\
ИНС  & искусственная нейронная сеть \\
ИС   & информационная система \\
ИТ   & информационные технологии \\
КИС  & корпоративная информационная система \\
НМ   & нечёткое множество \\
ОЗУ  & оперативное запоминающее устройство \\
ОС   & операционная система \\
ПО   & программное обеспечение \\
СИБ  & система информационной безопасности \\
СУБД & система управления базами данных \\
СЭД  & система электронного документооборота \\
ЭД   & электронный документ \\
ЭДО  & электронный документооборот \\
\end{tblr}

% Введение
\cchapter{ВВЕДЕНИЕ}                         % Заголовок
\addcontentsline{toc}{chapter}{ВВЕДЕНИЕ}    % Добавляем его в оглавление

Введение должно отражать: оценку современного состояния решаемой 
научно-технической проблемы, основание и исходные данные для разработки 
темы ВКР, обоснование необходимости её выполнения; описание цели 
и поставленных в работе задач. Во введении должны быть показаны: 
актуальность и новизна темы, связь данной работы с тематикой кафедры 
и с другими научно-исследовательскими работами.

% Разделы ВКР
\chapter{Название первой главы}\label{ch:1}

Текст первой главы~\cite{Temkin:2020,Rzazade:2023}.

\chapter{Название второй главы}\label{ch:2}

Текст второй главы


% Заключение
\cchapter{ЗАКЛЮЧЕНИЕ}                       % Заголовок
\addcontentsline{toc}{chapter}{ЗАКЛЮЧЕНИЕ}  % Добавляем его в оглавление

Заключение должно содержать:
\begin{itemize}
  \item краткие выводы по результатам выполненной ВКР или отдельных её этапов;
  \item оценку полноты решений поставленных задач;
  \item разработку рекомендаций и исходных данных по конкретному использованию результатов ВКР;
  \item результаты оценки технико-экономической эффективности внедрения (если имеет место);
  \item результаты оценки научно-технического уровня выполненной ВКР в сравнении с достижениями в этой области.
\end{itemize}


% Список литературы
\clearpage
\urlstyle{rm}                               % ссылки URL обычным шрифтом
\ifdefmacro{\microtypesetup}{\microtypesetup{protrusion=false}}{}
\insertbibliofull
\ifdefmacro{\microtypesetup}{\microtypesetup{protrusion=true}}{}
\urlstyle{tt}                               % возвращаем установки шрифта ссылок URL

\end{document}
